% Created 2024-07-08 Mon 22:16
% Intended LaTeX compiler: pdflatex
\documentclass[11pt]{article}
\usepackage[utf8]{inputenc}
\usepackage[T1]{fontenc}
\usepackage{ragged2e}
\usepackage{caladea}
\usepackage{graphicx}
\usepackage{longtable}
\usepackage{wrapfig}
\usepackage{rotating}
\usepackage[normalem]{ulem}
\usepackage{amsmath}
\usepackage{amssymb}
\usepackage{capt-of}
\usepackage{hyperref}
\usepackage{fancyhdr}
\title{Novena à Santa Bibiana}
 % \hypersetup{
 %  pdfauthor={},
 %  pdftitle={Novena a/à SANTO_NOME},
 %  pdfkeywords={},
 %  pdfsubject={},
 %  pdfcreator={Emacs 29.4 (Org mode 9.6.15)}, 
 %  pdflang={English}
 % }

\title{
  \includegraphics[scale=0.35]{./assets/imagem.jpg} \par
  NOVENA A SÃO FRANCISCO XAVIER}
\author{Garamog, Nina Freitas}
\date{24/11 - 03/12}
\renewcommand{\contentsname}{Sumário}

\begin{document}


\maketitle

\pagestyle{fancy}
  
\newpage

\tableofcontents

\centering
\vfill
Visite-nos no Telegram: \url{https://t.me/CotidieNovena}
\newpage


\section{História}\label{historia}


\begin{justify}

São Francisco Xavier nasceu nas terras de sua própria família denominada Xavier, Reino de Navarra, na península Ibérica, no dia 7 de Abril de 1506. Foi o filho caçula da família. Seu pai se chamava Juan de Jasso e era um aristocrático conselheiro do Rei de Navarra, João III. Sua mãe se chamava Maria de Azpilicueta y Xavier, era de família nobre também de Navarra.

Em 1512, o reino de navarra foi atacado por tropas castelhanas e aragonesas. Sua família lutou resistindo à invasão, mas perderam a luta em 1515. Francisco tinha quase nove anos. Seus irmãos foram presos e condenados a morrer depois de passarem um tempo na masmorra. Depois, porém, conseguiram ser libertados.

\subsection{Formação de São Francisco Xavier}

Neste período, aos 9 anos de idade, seu pai veio a falecer. Sua mãe envia-o, aos catorze anos, ao Colégio de Santa Bárbara, na cidade de Paris. Francisco se preparou para entrar na universidade, e terminou os estudos preliminares de literatura, humanidades e filosofia. Aprendeu a falar em francês, italiano e alemão. Passou a lecionar filosofia no Colégio de Beauvais. Conta-se que ele foi o campeão numa competição de salto em altura entre estudantes.

\subsection{Cofundador da Ordem dos Jesuítas}

Francisco dividia o quarto com um francês chamado Le Fèvre e um espanhol chamado Inácio de Loyola, futuro santo e fundador da ordem dos jesuítas. Eles criam um grupo com o nome de Societas Jesus (Sociedade de Jesus). Com mais quatro jovens, eles fundam a Companhia de Jesus, atualmente conhecida como Jesuítas, é a maior ordem religiosa do mundo. Eles fizeram voto de pobreza e foram reconhecidos pelo Papa em 1541. Francisco de Xavier foi ordenado sacerdote em Veneza, em 24 de junho de 1537.

\subsection{O missionário São Francisco Xavier}

Em setembro de 1543, São Francisco Xavier parte para a sua primeira missão. Foi para um lugar chamado Costa de Pescaria, que ficava no litoral sul da Índia. Nesse local o povo vivia da pesca. Porém, os hindus se tornavam inimigos daquele povo porque não aceitavam a pescaria que matava os peixes. Porém, os pescadores locais se identificaram de pronto com o cristianismo, pois esta religião aceitava sua profissão de pescadores. Eles se identificaram com os primeiros apóstolos de Jesus, que eram pescadores, e com um dos símbolos da nova religião, o peixe. O trabalho de Francisco de Xavier converteu o povo da Costa de Pescaria e provocou mudanças nas ilhas da Indonésia Oriental. Por isso, ele passou a ser chamado de o Apóstolo das Índias.
\subsection{São Francisco Xavier vai ao Japão}

Francisco chegou como missionário ao Japão em julho de 1549. O navio em que viajava, porém, aportou em Kagoshima, na ilha de kiushu, somente em agosto. Ele foi muito bem recebido e se hospedou na casa de Angiró até outubro do ano seguinte.

Depois, foi para Quioto, mas só foi autorizado a pregar em 1551. Como não falava a língua japonesa, teve que se contentar em ler o catecismo, que era traduzido pelo amigo Angiró. São Francisco Xavier, porém, foi perseverante e paciente. Alguns anos mais tarde, sua missão no Japão começou a dar frutos.

Ele conseguiu firmar congregações religiosas nas cidades de Bungo, Hirado e Yamaguchi. Xavier trabalhou lá durante mais dois anos. No seu ardor missionário, aprendeu a língua japonesa, na qual escreveu um livro que falava da criação do mundo e sobre a vida de Jesus Cristo. Depois ele foi substituído por outros padres jesuítas, os quais supervisionou durante um tempo. Então, regressou à Índia para fazer ainda inúmeros trabalhos de evangelização.
\subsection{Morte}

No dia 3 de dezembro de 1552, Francisco Xavier faleceu deitado numa esteira, com o crucifixo que o amigo Inácio de Loyola tinha lhe dado. Seu sepultamento foi em Sanchoão, mas seus restos mortais, que continuavam incorruptos, foram levados temporariamente à Igreja de São Paulo, em Malaca, em 1553. Em seguida, seus restos mortais são levados para a Basílica do Bom Jesus de Goa. Ele pode ser visto, ainda hoje, numa caixa de vidro e prata. As peregrinações ao local começaram em dezembro de 1637.

Um osso do braço direito de São Francisco Xavier está na igreja de São José, em Macau, na China, mantido em um relicário. A partir de sua morte, muitas igrejas foram construídas em sua homenagem.

\subsection{Milagres de São Francisco Xavier}

São Francisco Xavier tinha o dom de prever o futuro, ver fatos que aconteciam em outro lugar e acalmar tempestades. Na arte, sua imagem é associada a um caranguejo e um crucifixo, símbolo relacionado com um grande milagre. Aconteceu quando São Francisco navegava na região das ilhas Molucas, as quais ele evangelizou e onde viveu muitos anos. Durante um forte temporal, os marinheiros gritavam pela ajuda de Deus. Com calma, o santo mergulhou seu crucifixo de madeira no mar agitado. Na mesma hora o as águas se acalmaram e o crucifixo sumiu no mar. Ao desembarcarem na praia, os marinheiros, assombrados, viram um caranguejo sair da água com o crucifixo. Xavier o recolheu e o animal voltou para o mar. São Francisco Xavier nunca disse uma palavra sobre o acontecido.
\subsection{Devoção a São Francisco Xavier}

Francisco de Xavier foi beatificado em 25 de outubro de 1619, pelo Papa Paulo V, e, junto com Inácio de Loyola, foi canonizado em 1622, pelo Papa Gregório XV. Ele é considerado o protetor dos missionários. A Igreja celebra seu dia em 3 de dezembro.
\subsection{Oração a São Francisco Xavier}

Amabilíssimo e amantíssimo Santo, em união convosco adoro reverentemente a Divina Majestade e pelo muito que me regozijo dos especialíssimos dons da graça com que vos favoreceu durante a vossa vida mortal e pela glória que gozais agora, eu rendo-lhe afetuosíssimas graças e peço-lhe do fundo de minha alma e por vossa poderosa intercessão me conceda a graça importantíssima de viver e morrer santamente. E vos suplico também... (aqui o pedido especial) e se o que peço não convier à glória de Deus e ao proveito de minha alma, quero alcançar aquilo que a uma e outra seja mais conforme.

\end{justify}
Amém!

\subsection{Créditos }
\href{https://cruzterrasanta.com.br/historia-de-sao-francisco-xavier/141/102/}{Cruz Terra Santa}

\newpage
\section{Orações}\label{oracoes}


\subsection{Oração Inicial}

Oh! amabilíssimo e amantíssimo São Francisco Xavier, convosco
humildemente adoro a divina majestade, pois que me regozijo e lhe dou
infinitos louvores pelos singularíssimos dons de graças que vos concedeu
durante a vida, e de glória depois da morte e com todo coração vos peço
me alcanceis a preciosíssima graça de viver e morrer santamente.
Peço-vos também. (pede-se a graça desejada); e, se isso não é da maior
glória de Deus e maior bem da minha alma, alcançai-me o que mais
conforme for a uma e outra coisa assim seja.

\subsection{Ladainha}
\begin{justify}
+ Pai-Nosso, Ave-Maria e Glória. +

Senhor, \textbf{tende piedade}.

Cristo, \textbf{tende piedade}.

Senhor, \textbf{tende piedade}.

Cristo, \textbf{ouvi-nos}.

Cristo, \textbf{atendei-nos}.

Deus, o Pai do céu, \textbf{tende piedade de nós}.

Deus Filho, Redentor do mundo, \textbf{tende piedade de nós}.

Deus, o Espírito Santo, \textbf{tende piedade de nós}.

 Santíssima Trindade, um só Deus, \textbf{tende piedade de nós}.

Santa Maria, Virgem Mãe de Deus, \textbf{rogai por nós}.

Santo Inácio, fundador da Companhia de Jesus, \textbf{rogai por nós}.

São Francisco Xavier, a glória e o segundo pilar do Instituto Santo, \textbf{rogai por nós}.

Apóstolo das Índias e do Japão, \textbf{rogai por nós}.

Legado da Santa Sé Apostólica, \textbf{rogai por nós}.

Pregador da verdade e doutor das nações, \textbf{rogai por nós}.

Vaso de eleição, para levar o Nome de Jesus Cristo para os reis da terra,

\textbf{rogai por nós}.

Brilhando a luz para os que estavam sentados na sombra da morte, rogai

por nós.

Cheio de zelo ardente pela glória de Deus, \textbf{rogai por nós}.

Incansável propagador da fé cristã, \textbf{rogai por nós}.

Vigilante pastor de almas, \textbf{rogai por nós}.

Meditador constante sobre as coisas divinas, \textbf{rogai por nós}.

Fiel seguidor de Jesus Cristo, \textbf{rogai por nós}.

Amante ardente da pobreza evangélica, \textbf{rogai por nós}.

Observador perfeito da obediência religiosa, \textbf{rogai por nós}.

Abrasado com o fogo do Amor Divino, \textbf{rogai por nós}.

Generosamente desprezava todas as coisas terrenas, \textbf{rogai por nós}.

Guia dos poderosos, no caminho da perfeição, \textbf{rogai por nós}.

Modelo de homens apostólicos, \textbf{rogai por nós}.

Modelo de todas as virtudes, \textbf{rogai por nós}.

Luz dos infiéis e mestre dos fiéis, \textbf{rogai por nós}.

Anjo em vida e costumes, \textbf{rogai por nós}.

Patriarca de carinho e o cuidado do povo de Deus, \textbf{rogai por nós}.

Profeta poderoso em palavras e obras, \textbf{rogai por nós}.

Quem todas as nações e da Igreja têm associado a uma só voz...

...com o coro glorioso dos Apóstolos, \textbf{rogai por nós}.

Tu que foste adornado com a coroa dos castos, \textbf{rogai por nós}.

Quem aspirava a palma dos mártires, \textbf{rogai por nós}.

Confessor, em virtude da vida e da profissão, \textbf{rogai por nós}.

A quem os ventos e o mar obedeceu, \textbf{rogai por nós}.

Quem tomou de assalto as cidades que se revoltaram contra Jesus Cristo,

\textbf{rogai por nós}.

Que foste o terror dos exércitos de infiéis, \textbf{rogai por nós}.

Castigo de demônios e destruidor de ídolos, \textbf{rogai por nós}.

Poderosa defesa contra naufrágio, \textbf{rogai por nós}.

Pai dos pobres e refúgio dos miseráveis, \textbf{rogai por nós}.

Vista aos cegos e os coxos, \textbf{rogai por nós}.

Protetor em tempo de guerra, fome e peste, \textbf{rogai por nós}.

Maravilhoso trabalhador de milagres, \textbf{rogai por nós}.

Que foste dotado com o dom de línguas, \textbf{rogai por nós}.

Que foste dotado com o maravilhoso poder de ressuscitar os mortos,

\textbf{rogai por nós}.

Ressonante trombeta do Espírito Santo, \textbf{rogai por nós}.

Luz e glória do Oriente, \textbf{rogai por nós}.

Através da cruz, que tu tantas vezes levantaste entre os gentios, \textbf{rogai por nós}.

São Francisco Xavier, nós te rogamos, \textbf{ouvi-nos}.

Através da fé, que tu tão maravilhosamente propagava, nós te rogamos,

\textbf{ouvi-nos}.

Através de teus milagres e profecias, nós te rogamos, \textbf{ouvi-nos}.

Através dos perigos e naufrágios que tu suportou, nós te rogamos,

\textbf{ouvi-nos}.

Através das dores e trabalhos, no meio dos quais tu tão ardentemente exclamar:

"Ainda mais! Ainda mais!" nós te rogamos, \textbf{ouvi-nos}.

Através de teus arroubos celeste, no meio dos quais tu tão fervorosamente exclamava:

"Chega, chega, Senhor, o suficiente!" nós te rogamos, \textbf{ouvi-nos}.

Através da glória e felicidade que agora tu desfrutas no Céu, nós te rogamos, \textbf{ouvi-nos}.

Amigo do Noivo celeste, \textbf{intercedei por nós}.

São Francisco Xavier, amado de Deus e dos homens, \textbf{intercedei por nós}.

Cordeiro de Deus, que tirais os pecados do mundo, \textbf{Perdoai-nos, Senhor}.

Cordeiro de Deus, que tirais os pecados do mundo, \textbf{Atendei-nos, ó Senhor}.

Cordeiro de Deus, que tirais os pecados do mundo, \textbf{Tende piedade de nós.}

Cristo, ouvi-nos.

Cristo, atendei-nos.

Rogai por nós, São Francisco Xavier, Para que sejamos dignos das

promessas de Cristo

\end{justify}

\subsection{Créditos:}
Nada Te Espante + Devoção de São Francisco Xavier


\end{document}
