\documentclass[18pt]{article}
\usepackage[utf8]{inputenc}
\usepackage[T1]{fontenc}
\usepackage{ragged2e}
\usepackage{caladea}
\usepackage{graphicx}
\usepackage{longtable}
\usepackage{wrapfig}
\usepackage{rotating}
\usepackage{epigraph}
\usepackage[normalem]{ulem}
\usepackage{hyperref}
\usepackage{amsmath}
\usepackage{amssymb}
\usepackage{capt-of}
\usepackage{hyperref}
\usepackage{fancyhdr}

\title{
 \includegraphics[scale=0.7, trim={10cm, 0, 10cm, 0}]{./assets/imagem.jpg}
  \par
   NOVENA A SÃO SEVERINO }
\author{Garamog, Nina Freitas}
\date{Início da Novena: 30/12 - Data Litúrgica: 08/01 }

% Comando para fazer "Sumário" não aparecer no Sumário.
\renewcommand{\contentsname}{Sumário}
\begin{document}
\maketitle

\thispagestyle{empty} %zera a primeira página


\pagestyle{fancy}
\fancyhf{} % clear existing header/footer entries
\fancyfoot[LO, CE]{
  \includegraphics[scale=0.2]{./assets/cross.png} São Severino, rogai por nós!
}
% Place Page X of Y on the right-hand
% side of the footer
\fancyfoot[R]{\thepage}

\newpage

\tableofcontents

\centering
\vfill
Visite-nos no Telegram: \url{https://t.me/CotidieNovena}
\newpage

\newpage


\begin{justify}

 \begin{center}
  \section{História}\label{sec:História} % (fold)
 \end{center}

 \subsection{Origens}
O discípulo e biógrafo de Santo Severino, chamado Eugípio, afirma que ele teria nascido no século V, em 410, em Roma, na Itália. Era de família nobre e muito rica. Tinha ótima educação, falava fluentemente bem o latim, e era, ao mesmo tempo, uma pessoa humilde e profundamente caridosa. Severino tinha o dom da cura, da profecia e do conselho. Ele vivia fielmente os votos sacerdotais, fazia muita penitência e rezava sempre ao divino Espírito Santo.


\subsection{Tempo de guerra}
No século V o Ocidente sofreu várias invasões de povos, como, godos, ostrogodos, visigodos, burgúndio e vândalos. As vítimas dessas guerras sucessivas encontravam segurança e abrigo somente com os padres e religiosos da Igreja, e São Severino aproveitava para oferecer, além das necessidades básicas dos refugiados, uma boa evangelização cristã.

\subsection{Viagens}
Em 454, São Severino foi enviado à Nórica (atualmente da Áustria à Baviera, na Alemanha) e à Pomonia, na Áustria, às margens do rio Danúbio. O local tornou-se um ponto estratégico para que ele acolhesse a população sempre ameaçada, pregando, também, para bárbaros e pagãos. Em 455, esteve em Melk e, depois, em Ostembur, onde ele passou a viver numa cabana, entregando-se à penitência e à oração.

\subsection{Fundador e profeta}
Esse ministério itinerante de São Severino chegou a várias cidades, nas quais ele fundou alguns mosteiros. Com suas profecias, acertava as datas das invasões e avisava as comunidades do perigo eminente. Em Asturis, ele profetizou a invasão e mortes comandadas por Átila, rei dos hunos, habitantes da Hungria. As pessoas, porém, não lhe deram atenção e zombavam dele. Pouco depois da partida de Severino a cidade foi invadida, saqueada e os habitantes foram mortos. Chegando a Comagaris e depois Comagene, lugares dominados por invasores, socorria as pessoas, conseguindo o respeito inclusive dos inimigos.

\subsection{Graças e milagres}
A história de São Severino está repleta de graças e milagres. Ele predisse a data de sua morte e que sua Ordem Religiosa seria expulsa da região do Danúbio. Ele morreu no dia 8 de janeiro de 482, depois de pronunciar uma frase do Salmo 150: "Todo ser que tem vida, a deve ao Senhor".

\subsection{Veneração}
São Severino é muito venerado na Alemanha e na Áustria. Seu túmulo está em Nápoles, Itália, na igreja dos beneditinos. Ele é considerado padroeiro dos prisioneiros e dos pobres.


\end{justify}

%%%%%%%%%%%%%%%%%%%%%%%%%%%%%%%%%%%%% Orações  %%%%%%%%%%%%%%%%%%%%%%%%%%%%%%%%%%%%%%%%%%%

\section{Oração}\label{sec:Oração} % (fold)

Ó glorioso São Severino, que foste enviado para ser luz e conforto nas terras da Nórica, onde tua presença trouxe paz aos aflitos e proteção aos desamparados, acolhe hoje a nossa súplica. Inspirados por tua vida de desprendimento e caridade, clamamos por tua intercessão para que também nós possamos enfrentar as provações com firmeza e generosidade.

Recordamos como, em tempos de angústia e escassez, ofereceste abrigo e alimento aos necessitados, e nos espelhamos em tua coragem ao enfrentar as adversidades para honrar o nome de Cristo. Concede-nos, ó servo fiel, a mesma ousadia em proclamar a verdade da fé e em viver a caridade que o Senhor nos ensina, buscando em cada dia o fortalecimento de nossa entrega a Ele.

Pedimos que intercedas junto a Deus para que possamos crescer em compaixão e desprendimento, abandonando o egoísmo e servindo com humildade. Que a paz que semeaste em meio ao povo floresça em nossos corações e que, com tua intercessão, sejamos fortalecidos na caminhada cristã.

\textbf{Pai Nosso, Ave Maria, Glória ao Pai.}

\subsection*{Créditos:}
\href{https://cruzterrasanta.com.br/historia-de-sao-severino/335/102/}{Cruz Terra Santa}



\end{document}
